\documentclass[draft,color,12pt,letterpaper,sans]{moderncv}
\moderncvtheme[green]{casual}
\usepackage[utf8]{inputenc}
%\usepackage{fontawesome}
\usepackage[margin=1in,bottom=2in,left=0.7in]{geometry}
\setlength{\hintscolumnwidth}{7em}
\recomputelengths

\title {Desarrollador Senior Full Stack y SysAdmin Linux}
\firstname{Aaron} % Your first name
\familyname{Cruces Cáceres} % Your last name
% \address{Rolando Biaggini 631, Antofagasta}{Postal: 1310091}{Chile}
\mobile{+56-9-7423-8157}
\email{aaroncruces@gmail.com}
\social[linkedin]{https://www.linkedin.com/in/aaroncruces/}
\social[github]{https://github.com/aaroncruces}

% Si se usa "modern", no usar los fontawesome. intentar revisar la documentacion de "\addtofooter" en texdoc y "modernvcsylecasual.sty""

\nopagenumbers{}
\begin{document}
\makecvtitle % Obligatorio

\section{Experiencia Laboral Profesional}

\cventry {2023: Julio - Actualidad}
{Ingeniero Senior DevOps}
{Centro Educacion Industrial y Minero}
{\newline Presencial, Antofagasta}
{\newline Tecnologías: Microsoft Azure, Debian Linux, Docker, SMB Samba, Gitlab, Ansible}
{Implementación de despliegues automatizados de arquitectura de servicios utilizando scripts Ansible y Bash versionados con GIT.\newline Creación de sistemas de integración continua mediante instancias de Gitlab runners. Gitlab, database y otros servicios automatizados mediante lanzamientos automáticos de contenedores Docker utilizando Ansible.\newline}


\cventry {2023: Marzo - Junio}
{Profesor de Informática}
{Instituto Profesional IPG}
{\newline Presencial}
{\newline Tecnologías: Microsoft Office}
{Enseñanza y entrenamiento práctico en el uso de tecnologías utilizadas para cada carrera asignada. Predominantemente el uso de Microsoft Office, dada la naturaleza de la carreras cuyo curso ha sido impartido.\newline}

\cventry {2023: Marzo - Junio}
{Ingeniero y Técnico Informático}
{Instituto Profesional IPG}
{\newline Presencial}
{\newline Tecnologías: Microsoft Excel, SMB Samba, FreeBSD, Linux (Arch y Debian)}
{Implementación de soluciones empresariales en oficina de  admisión estudiantil. Instalación de carpetas compartidas y de respaldo mediante uso de servidor dedicado FreeBSD con SMB Samba y ZFS (mirror). Programación de soluciones para manipulación de datos en Microsoft Excel mediante Typescript. Análisis de arquitectura de redes en escuela colaboradora Julia Herrera Varas.\newline}

\cventry {Junio 2022 - Marzo 2023}
{Desarrollador Web React e Ingeniero Servicios Linux}
{Tecnojosue}
{\newline Desarrollo Remoto, Concepcion}
{\newline Tecnologías:  ReactJS, TypeScript, Prisma ORM, ExpressJS, Linux (Arch y Debian), scripting Bash (unix) y BAT (Windows), servicios de archivos y web (SMB, Cloudflare DynDNS, Portainer-Docker)}
{Desarrollo de aplicación web para ingreso y gestión de inventariado mediante ReactJS y TypeScript. Implementación de aplicaciones ”self hosted” en servidores Linux mediante uso de cron-jobs, daemons y docker.\newline}

\cventry {2021: Septiembre-Octubre}
{Desarrollador Full Stack PHP}
{Te Para Tres, Santiago}
{\newline Desarrollo remoto}
{\newline Tecnologías: PHP, VSCode, GIT, Debugging con XDebug, Laravel, FacturaScripts, trello}
{Mantención y análisis de sistemas legacy creados con Laravel. Integración y desarrollo de sistemas preexistentes y nuevos a plugin FacturaScripts en PHP.\newline}

\cventry {2021: Mayo-Agosto}
{Desarrollador Full Stack Freelance MERN}
{TeslaDelta, Concepción}
{\newline Desarrollo remoto}
{\newline Tecnologías: TypeScript, ReactJS, ReduxJS, Webpack, NodeJS, NPM, ExpressJS, MongoDB, Bootstrap, Sass, Figma, SSH remoto con Ngrok}
{Creación de plataforma web para la el ingreso, modificación y actualización con código de barras y QR de productos en inventariado.\newline}

\cventry {2020: Septiembre-Diciembre}
{Ingeniero informático}
{Moldava; Intergrade}
{Municipalidad de Lo Barnechea}
{\newline Tecnologías: Excel macros VBA, Aida64, Microsoft SMB, Acronis True Image}
{Automatización del scrapping y parsing de reportes de Aida64 mediante Excel VBA; Migración de Sistemas Operativos; Instalación y administración de software y hardware.\newline}


\cventry {2020: Febrero-Abril}
{Ingeniero informático}
{Intergrade; Just Services}
{Dirección de presupuestos. Ministerio de hacienda}
{\newline Tecnologías: Microsoft VBscript, Microsoft SMB, Acronis True Image}
{Automatización en VBscript de procesos de migración de Sistemas Operativos; Instalación y administración de software y hardware.\newline}

\cventry {2019: Octubre-Noviembre}
{Soporte técnico informático}
{Advanced Computing Technologies}
{Hospital de Urgencia y Asistencia Pública. Santiago}
{}
{Subcontratación para la realización de servicios de soporte informático en Hospital de Urgencia. Comunicación y coordinación entre personal médico y técnico informático.\newline}

\cventry {2019: Agosto-Octubre}
{Ingeniero de sistemas linux, Técnico Informático}
{La Casa Del Notebook}
{Calle San Diego, Santiago Centro}
{\newline Tecnologías: Clonezilla, Linux, MercadoLibre, Microsoft Windows, Microsoft Excel}
{Administración, e instalación de sistemas Linux. Técnico encargado del área de software, mantenimiento de hardware y control de inventario. Mantenimiento, reparación y venta de equipos informáticos.\newline} 

\cventry {Marzo 2017 - Noviembre 2018 }
{Ingeniero de Servicios T.I.}
{Ecacom S.P.A.}
{Molycop; Terquim. }
{\newline Tecnologías: OpenWRT, Microsoft Office, Microsoft Windows}
{Subcontratación de parte de empresa Ecacom para prestación de servicios de diversa índole. Incluyendo, relaciones con cliente, administración de equipos, mantenimiento, instalación y configuración de sistemas de red (OpenWRT).\newline}

\cventry {2016: Julio-Noviembre}
{Desarrollador Full Stack MEAN}
{Escuela Netland School.}
{Antofagasta}
{\newline Tecnologías: Jetbrains WebStorm, AngularJS, Bootstrap, HTMl, CSS, JavaScript, NodeJS, ExpressJS, MongoDB, Yarn, Gulp, Yeoman, Trello}
{Construcción de plataforma informática para conexión de alumnos y docentes. Construyendo servicio web y aplicación movil Android. Proyecto realizado con equipo de 3 ingenieros. \newline}

\cventry {2016: Marzo-Junio}
{Desarrollador Frontend PHP}
{Escuela Netland School}
{Antofagasta}
{\newline Tecnologías: Jetbrains PhpStorm, PHP, Wordpress, Cpanel, HTML, CSS, JavaScript}
{Construcción de plataforma web para la comunicación de actividades de clase de educación física. Uso de plataforma CMS Wordpress para el dise\~no y administración de actividades.\newline}

\cventry {2016: Enero-Marzo}
{Desarrollador JAVA}
{SEREMI de educación de la región del Biobío}
{Concepción}
{\newline Tecnologías: Java, Eclipse, Swing, Apache POI, Microsof Excel}
{Creación de aplicación de formulario para cálculo, automatización y despacho de viáticos a trabajadores.\newline}


\section{Experiencia proyectos personales}

\cventry {}
{Setup musical linux}
{Proyecto personal para músico, Sintetizador MIDI}
{}
{\newline Tecnologías: Linux, SSH, TMUX, MIDI, Jack Audio Connection Kit, KXStudio Cadence, VST/VSTi Linux Plugins}
{Setup para músico con la necesidad sintetizar con notebook con mínimos requerimientos, con cero configuración de su parte.\newline}
\cventry {}
{Setup musical windows}
{Proyecto personal para músico, Mandos MIDI}
{}
{\newline Tecnologías: FLStudio, ASIO4All, MJOY, MidiYokeNT}
{Setup para músico con la necesidad de controlar setup MIDI con mandos de videojuegos\newline}


\cventry {}
{Impresion 3D Litófanos}
{Regalo de cumpleaños para abuela}
{}
{\newline Tecnologías: Autodesk Fusion360, GIMP, LuBan3D Lithophane Creator, Autodesk MeshMixer, PrusaSlicer, Prusa MK2s 3D Printer}
{Creación de litófanos para lámpara con fotos de nietos. Construcción de diseño paramétrico de jaula con Fusion360.
Transformación de fotografías a litófano con la cadena GIMP \faArrowRight LuBan \faArrowRight MeshMixer. Fabricación con impresora 3D Prusa MK2s \newline}



\section{Antedentes Académicos}
\cventry{2009-2017}
{Ingeniería Civil en Computación e Informática}
{Universidad Católica del Norte}
{Antofagasta, Chile}
{\textit{Licenciado en ciencias de la ingeniería}}
{}


\section{Conocimientos}
\begin{cvcolumns}
	\cvcolumn{Lenguajes y Frameworks}
	{
		\begin{itemize}
			\item
				ReactJS
			\item
				ReduxJS
			\item
				ExpressJS
			\item
				Webpack
			\item
				TypeScript
			\item
				MongoDB, Mongoose
			\item
				SQL
			\item
				HTML
			\item
				CSS
			\item
				VBA
			\item	
				VBScript
			\item
				JAVA		
			\item
				Bash
			\item
				C++
			\item
				VB.net
			\item
				LaTeX
		\end{itemize}
	}
	

\cvcolumn{Tecnologías}
{
	\begin{itemize}
		\item
			GIT
		\item
			Visual Studio Code
		\item
			Jetbrains Webstorm, PhpStorm
		\item
			Eclipse
		\item
			SysAdmin Linux
		\item
			Windows
		\item
			Microsoft Office con VBA
		\item
			Fusion 360
		\item
			VMWare y VirtualBox
		\item
			OpenWRT y ddWrt
		\item
			GIMP
		\item
			Impresion 3D FFF
		\item
			MIDI
		\item	
			Vim
	\end{itemize}
}

\end{cvcolumns}



\newpage

\section{Idioma Inglés}
\cvlanguage{Escritura}
{Medio Alto}
{Escritura Fluida.}
\cvlanguage{Lectura}
{Alto}
{Lectura fluida.}
\cvlanguage{Hablado}
{Medio Alto}
{Conversación a ritmo normal.}
\cvlanguage{Escucha}
{Alto}
{Sin problemas para entender videos, podcast, conferencias en audio nativo}

%\newpage


\end{document}
