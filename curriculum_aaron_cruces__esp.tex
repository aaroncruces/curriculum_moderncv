\documentclass[12pt,letterpaper,sans]{moderncv}
\moderncvtheme[green]{casual}
\usepackage[utf8]{inputenc}
\usepackage[margin=1in]{geometry}
\setlength{\hintscolumnwidth}{7em}
\recomputelengths

\firstname{Aaron}
\familyname{Cruces Cáceres}
\mobile{+56-9-7423-8157}
\email{aaroncruces@gmail.com}
\social[linkedin]{https://www.linkedin.com/in/aaroncruces/}
\social[github]{https://github.com/aaroncruces}
\title {Ingeniero Senior Cloud, DevOps, SysAdmin Linux y Desarrollador Full Stack}

\begin{document}
\makecvtitle

\section{Experiencia Laboral Profesional}

\cventry{Julio 2023 -- Actualidad}
{Ingeniero Senior Infraestructura Cloud y DevOps}
{Centro Educación Industrial y Minero, CEIM}
{\newline Presencial, Antofagasta}
{\newline Tecnologías: Microsoft Azure, Debian Linux, Docker, Samba, GitLab, Ansible, Nginx.}
{
  \begin{itemize}
    \item Despliegue de servicios App Services y máquinas virtuales utilizando Microsoft Azure y servidores locales con Proxmox.
    \item Gestión de sistemas operativos Windows Server 2022 con virtualización anidada, Alma Linux y Debian.
    \item Configuración de plataformas Docker con Portainer, Nginx, IIS y servicios \textit{standalone}.
    \item Creación de un sistema de almacenamiento con discos duros como carpeta compartida para desarrolladores, utilizando ZFS, TrueNAS, Samba y virtualización con Proxmox (PCI Passthrough).
    \item Implementación de un gestor de contraseñas centralizado Bitwarden (Vaultwarden) para la gestión de información crítica.
    \item Automatización de despliegues con scripts de Ansible y Bash, versionados con Git.
    \item Configuración de sistemas de integración continua mediante GitLab Runners, con despliegues automáticos de contenedores Docker utilizando Ansible.
  \end{itemize}
}
\vspace{0.5em}

\cventry{Marzo -- Junio 2023}
{Profesor de Informática}
{Instituto Profesional IPG}
{\newline Presencial}
{\newline Tecnologías: Microsoft Office.}
{
  \begin{itemize}
    \item Capacitación práctica en el uso de Microsoft Office para estudiantes de diversas carreras.
  \end{itemize}
}
\vspace{0.5em}

\cventry{Marzo -- Junio 2023}
{Ingeniero y Técnico Informático}
{Instituto Profesional IPG}
{\newline Presencial}
{\newline Tecnologías: Microsoft Excel, Samba, FreeBSD, Linux (Arch, Debian).}
{
  \begin{itemize}
    \item Implementación de soluciones empresariales en la oficina de admisión estudiantil, incluyendo carpetas compartidas y respaldos mediante un servidor FreeBSD con Samba y ZFS (mirror).
    \item Programación de soluciones para manipulación de datos en Microsoft Excel utilizando TypeScript.
    \item Análisis y optimización de la arquitectura de redes en la escuela colaboradora Julia Herrera Varas.
  \end{itemize}
}
\vspace{0.5em}

\cventry{Junio 2022 -- Marzo 2023}
{Desarrollador Web React e Ingeniero Servicios Linux}
{Tecnojosue}
{\newline Remoto, Concepción}
{\newline Tecnologías: ReactJS, TypeScript, Prisma ORM, ExpressJS, Linux (Arch, Debian), Bash, BAT, Samba, Cloudflare DynDNS, Portainer-Docker.}
{
  \begin{itemize}
    \item Desarrollo de una aplicación web para gestión de inventarios utilizando ReactJS y TypeScript.
    \item Implementación de aplicaciones \textit{self-hosted} en servidores Linux mediante \textit{cron-jobs}, \textit{daemons} y Docker.
  \end{itemize}
}
\vspace{0.5em}

\cventry{Septiembre -- Octubre 2021}
{Desarrollador Full Stack PHP}
{Te Para Tres}
{\newline Remoto, Santiago}
{\newline Tecnologías: PHP, VSCode, Git, XDebug, Laravel, FacturaScripts, Trello.}
{
  \begin{itemize}
    \item Mantenimiento y análisis de sistemas \textit{legacy} desarrollados con Laravel.
    \item Integración y desarrollo de sistemas nuevos y preexistentes en el plugin FacturaScripts con PHP.
  \end{itemize}
}
\vspace{0.5em}

\cventry{Mayo -- Agosto 2021}
{Desarrollador Full Stack Freelance MERN}
{TeslaDelta}
{\newline Remoto, Concepción}
{\newline Tecnologías: TypeScript, ReactJS, ReduxJS, Webpack, NodeJS, ExpressJS, MongoDB, Bootstrap, Sass, Figma, SSH, Ngrok.}
{
  \begin{itemize}
    \item Creación de una plataforma web para ingreso, modificación y actualización de productos en inventarios con códigos de barras y QR.
  \end{itemize}
}
\vspace{0.5em}

\cventry{Septiembre -- Diciembre 2020}
{Ingeniero Informático}
{Moldava; Intergrade}
{\newline Municipalidad de Lo Barnechea}
{\newline Tecnologías: Excel VBA, Aida64, Samba, Acronis True Image.}
{
  \begin{itemize}
    \item Automatización de \textit{scraping} y \textit{parsing} de reportes de Aida64 mediante Excel VBA.
    \item Migración de sistemas operativos, instalación y administración de software y hardware.
  \end{itemize}
}
\vspace{0.5em}

\cventry{Febrero -- Abril 2020}
{Ingeniero Informático}
{Intergrade; Just Services}
{\newline Dirección de Presupuestos, Ministerio de Hacienda}
{\newline Tecnologías: VBScript, Samba, Acronis True Image.}
{
  \begin{itemize}
    \item Automatización de procesos de migración de sistemas operativos utilizando VBScript.
    \item Instalación y administración de software y hardware.
  \end{itemize}
}
\vspace{0.5em}

\cventry{Octubre -- Noviembre 2019}
{Soporte Técnico Informático}
{Advanced Computing Technologies}
{\newline Hospital de Urgencia y Asistencia Pública, Santiago}
{\newline Tecnologías: Ninguna específica.}
{
  \begin{itemize}
    \item Prestación de servicios de soporte informático, coordinando entre personal médico y técnico.
  \end{itemize}
}
\vspace{0.5em}

\cventry{Agosto -- Octubre 2019}
{Ingeniero de Sistemas Linux, Técnico Informático}
{La Casa Del Notebook}
{\newline Santiago Centro}
{\newline Tecnologías: Clonezilla, Linux, MercadoLibre, Windows, Excel.}
{
  \begin{itemize}
    \item Administración e instalación de sistemas Linux, mantenimiento de hardware y control de inventarios.
    \item Mantenimiento, reparación y venta de equipos informáticos.
  \end{itemize}
}
\vspace{0.5em}

\cventry{Marzo 2017 -- Noviembre 2018}
{Ingeniero de Servicios T.I.}
{Ecacom S.P.A.}
{\newline Molycop; Terquim}
{\newline Tecnologías: OpenWRT, Microsoft Office, Windows.}
{
  \begin{itemize}
    \item Prestación de servicios de soporte, incluyendo relaciones con clientes, administración de equipos y configuración de redes con OpenWRT.
  \end{itemize}
}
\vspace{0.5em}

\cventry{Julio -- Noviembre 2016}
{Desarrollador Full Stack MEAN}
{Escuela Netland School}
{\newline Antofagasta}
{\newline Tecnologías: JetBrains WebStorm, AngularJS, Bootstrap, HTML, CSS, JavaScript, NodeJS, ExpressJS, MongoDB, Yarn, Gulp, Yeoman, Trello.}
{
  \begin{itemize}
    \item Desarrollo de una plataforma web y aplicación móvil Android para conectar alumnos y docentes, en un equipo de tres ingenieros.
  \end{itemize}
}
\vspace{0.5em}

\cventry{Marzo -- Junio 2016}
{Desarrollador Frontend PHP}
{Escuela Netland School}
{\newline Antofagasta}
{\newline Tecnologías: JetBrains PhpStorm, PHP, WordPress, cPanel, HTML, CSS, JavaScript.}
{
  \begin{itemize}
    \item Creación de una plataforma web para la comunicación de actividades de educación física, utilizando WordPress como CMS.
  \end{itemize}
}
\vspace{0.5em}

\cventry{Enero -- Marzo 2016}
{Desarrollador Java}
{SEREMI de Educación, Región del Biobío}
{\newline Concepción}
{\newline Tecnologías: Java, Eclipse, Swing, Apache POI, Excel.}
{
  \begin{itemize}
    \item Desarrollo de una aplicación de formulario para cálculo, automatización y despacho de viáticos.
  \end{itemize}
}
\vspace{0.5em}

\newpage

\section{Experiencia en Proyectos Personales}

\cventry{}
{"Self-hosted Homelab"}
{Sistema de alojamiento y publicación de sitios web personales}
{\newline Tecnologías: Debian, Proxmox, SSH, Docker, Portainer, Authentik, ZFS, ZSH, Bash, MariaDB, PostgreSQL}
{\newline}
{
  \begin{itemize}
    \item Creación de arreglos de discos duros con controladores HBA SAS y ZFS para \textit{backups} y servicios \textit{self-hosted}.
    \item Despliegue de un gestor de contraseñas Bitwarden para manejo centralizado de claves.
    \item Configuración de servicios web con Portainer, protegidos con Authentik y publicados con Traefik.
    \item Gestión de control de versiones mediante Gitea.
    \item Despliegue de un servidor multimedia con Jellyfin y Nvidia Container Runtime.
    \item Gestión de bases de datos para contenedores con Docker Networks, PostgreSQL y MariaDB.
    \item Puesta a disposición de documentos y fotos mediante NextCloud y Samba, con deduplicación utilizando \textit{rdfind}, \textit{rsync} y \textit{scripting} en Bash.
    \item Automatización de despliegues con Ansible, GNU Stow, \textit{rsync} y \textit{scripting} en Bash.
  \end{itemize}
}
\vspace{0.5em}

\cventry{}
{Setup Musical Linux}
{Sintetizador MIDI para músicos}
{\newline Tecnologías: Linux, SSH, TMUX, MIDI, Jack Audio Connection Kit, KXStudio Cadence, VST/VSTi Linux Plugins}
{\newline}
{
  \begin{itemize}
    \item Configuración de un entorno Linux para sintetizar música con mínimos requerimientos y cero configuración por parte del usuario.
  \end{itemize}
}
\vspace{0.5em}

\cventry{}
{Setup Musical Windows}
{Control de mandos MIDI}
{\newline Tecnologías: FL Studio, ASIO4All, MJOY, MidiYokeNT}
{\newline}
{
  \begin{itemize}
    \item Configuración de un entorno Windows para controlar un setup MIDI con mandos de videojuegos.
  \end{itemize}
}
\vspace{0.5em}

\cventry{}
{Impresión 3D Litófanos}
{Regalo de cumpleaños}
{\newline Tecnologías: Autodesk Fusion 360, GIMP, LuBan3D Lithophane Creator, Autodesk MeshMixer, PrusaSlicer, Prusa MK2s}
{\newline}
{
  \begin{itemize}
    \item Creación de litófanos para una lámpara con fotos familiares, utilizando Fusion 360 para el diseño paramétrico y una cadena de herramientas (GIMP, LuBan, MeshMixer) para la transformación de imágenes, fabricados con una impresora 3D Prusa MK2s.
  \end{itemize}
}
\vspace{0.5em}

\newpage

\section{Antecedentes Académicos}
\cventry{2009--2017}
{Ingeniería Civil en Computación e Informática}
{Universidad Católica del Norte}
{\newline Antofagasta, Chile}
{\newline Licenciado en Ciencias de la Ingeniería.}
{}

\section{Conocimientos}
\begin{cvcolumns}
  \cvcolumn{Lenguajes y Frameworks}
  {
    \begin{itemize}
      \item Bash
      \item ExpressJS
      \item JavaScript
      \item NodeJS
      \item PHP
      \item Python
      \item ReactJS
      \item SQL
      \item TypeScript
      \item ZSH
    \end{itemize}
  }
  \cvcolumn{Tecnologías}
  {
    \begin{itemize}
      \item Ansible
      \item Docker
      \item Git
      \item GitLab
      \item Linux (Debian)
      \item Microsoft Azure
      \item Nginx
      \item Proxmox
      \item Samba
      \item ZFS
    \end{itemize}
  }
\end{cvcolumns}

\section{Idioma Inglés}
\cvlanguage{Escritura}{B2}{Escritura fluida}
\cvlanguage{Lectura}{C1}{Lectura fluida}
\cvlanguage{Hablado}{B2}{Conversación a ritmo normal}
\cvlanguage{Escucha}{C1}{Comprensión de videos, podcasts y conferencias en audio nativo}

\end{document}